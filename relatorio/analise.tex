Agora irei descrever a analise experimental, na qual eu fiz de duas formas.
\begin{enumerate}
    \item Grafica: Fiz experimentos com matrizes quadradas de lado no intervalo $[10^{1}, 10^{3}]$, no qual é feito samples igualmente espaçados. E assim gerei graficos a partir disso.
    \item Tabular: Fiz experimentos com matrizes quadradas de lado no intervalo $[10^{1}, 10^{6}]$, para cada potencia de 10, é estimado o tempo medio  
\end{enumerate}

Estou me referindo por TableMatrix, como a implementação que faz operações com uma representação de linhas e vetores na memoria. 

\section{Comparação}

Para todas as operações que eram para ter uma estimativa de tempo constante, apesar de ser estimativamente constante, há um crescimento mesmo em TableMatrix que faz acesso direto a memoria. 
Suponho que o problema disso, seja alocamento e desalocamento de memoria nos brenchmarks que deve estar acontecendo.  Por causa que as matrizes são criadas e logo após não usadas nunca mais, então logo após de ter sido aplicado a operação o programa já libera a memoria. 

Para TableMatrix, só foi possivel fazer samples até $10^3$, por causa  que além disso, a há estouro de memoria.

\subsection{Get}
Na leitura de valores, é perceptível que para hashmap é perceptivel que é o que tem a melhor performance que todos os outros, tendo tempo aproximadamente constante em nanosegundos para todos valores.

Provavelmente em TableMatrix há uma perda de desempenho de memoria da memoria com o crescimento da matriz por causa de paginação da memoria.

\begin{tabular}{c c c c c c}
\toprule
Tamanho & Ocupação & population & HashMapMatrix & TableMatrix & TreeMatrix \\
\midrule
$10^1$x$10^1$ & 1\% & 1 & 161.600 ns & 314.000 ns & 100.000 ns \\
$10^1$x$10^1$ & 5\% & 5 & 152.800 ns & 264.850 ns & 111.250 ns \\
$10^1$x$10^1$ & 10\% & 10 & 161.550 ns & 231.300 ns & 146.700 ns \\
$10^1$x$10^1$ & 20\% & 20 & 154.150 ns & 222.750 ns & 212.850 ns \\
$10^2$x$10^2$ & 1\% & 100 & 207.950 ns & 2.588 µs & 1.138 µs \\
$10^2$x$10^2$ & 5\% & 500 & 187.700 ns & 2.539 µs & 2.115 µs \\
$10^2$x$10^2$ & 10\% & 1000 & 142.000 ns & 2.740 µs & 4.144 µs \\
$10^2$x$10^2$ & 20\% & 2000 & 233.450 ns & 2.488 µs & 13.849 µs \\
$10^3$x$10^3$ & 1\% & 10000 & 635.300 ns & 80.230 µs & 65.414 µs \\
$10^3$x$10^3$ & 5\% & 50000 & 858.350 ns & 89.192 µs & 223.284 µs \\
$10^3$x$10^3$ & 10\% & 100000 & 844.100 ns & 91.729 µs & 672.044 µs \\
$10^3$x$10^3$ & 20\% & 200000 & 717.250 ns & 97.935 µs & 1.295 ms \\
$10^4$x$10^4$ & 0.0001\% & 100 & 146.550 ns &  & 545.250 ns \\
$10^4$x$10^4$ & 0.001\% & 1000 & 135.200 ns &  & 4.324 µs \\
$10^4$x$10^4$ & 0.01\% & 10000 & 302.050 ns &  & 44.709 µs \\
$10^5$x$10^5$ & 1e-05\% & 1000 & 214.450 ns &  & 6.978 µs \\
$10^5$x$10^5$ & 0.0001\% & 10000 & 498.400 ns &  & 57.752 µs \\
$10^5$x$10^5$ & 0.001\% & 100000 & 745.450 ns &  & 648.069 µs \\
$10^6$x$10^6$ & 1e-06\% & 10000 & 589.350 ns &  & 43.636 µs \\
$10^6$x$10^6$ & 1e-05\% & 100000 & 1.080 µs &  & 526.994 µs \\
$10^6$x$10^6$ & 0.0001\% & 1000000 & 4.596 ms &  & 5.626 ms \\
\bottomrule
\end{tabular}

\pagebreak

\subsection{Set}
Possui as mesmas características que Get.


\begin{tabular}{c c c c c c}
\toprule
Tamanho & Ocupação & population & HashMapMatrix & TableMatrix & TreeMatrix \\
\midrule
$10^1$x$10^1$ & 1\% & 1 & 168.000 ns & 286.450 ns & 663.250 ns \\
$10^1$x$10^1$ & 5\% & 5 & 137.250 ns & 238.650 ns & 118.700 ns \\
$10^1$x$10^1$ & 10\% & 10 & 150.500 ns & 232.150 ns & 186.900 ns \\
$10^1$x$10^1$ & 20\% & 20 & 149.100 ns & 300.750 ns & 383.150 ns \\
$10^2$x$10^2$ & 1\% & 100 & 198.050 ns & 3.444 µs & 921.800 ns \\
$10^2$x$10^2$ & 5\% & 500 & 145.250 ns & 2.602 µs & 2.817 µs \\
$10^2$x$10^2$ & 10\% & 1000 & 142.500 ns & 3.583 µs & 6.031 µs \\
$10^2$x$10^2$ & 20\% & 2000 & 199.050 ns & 2.659 µs & 11.045 µs \\
$10^3$x$10^3$ & 1\% & 10000 & 756.550 ns & 80.424 µs & 46.669 µs \\
$10^3$x$10^3$ & 5\% & 50000 & 777.250 ns & 102.121 µs & 231.141 µs \\
$10^3$x$10^3$ & 10\% & 100000 & 846.000 ns & 94.056 µs & 525.222 µs \\
$10^3$x$10^3$ & 20\% & 200000 & 753.150 ns & 102.774 µs & 1.107 ms \\
$10^4$x$10^4$ & 0.0001\% & 100 & 147.700 ns &  & 1.195 µs \\
$10^4$x$10^4$ & 0.001\% & 1000 & 207.500 ns &  & 8.130 µs \\
$10^4$x$10^4$ & 0.01\% & 10000 & 251.600 ns &  & 45.633 µs \\
$10^5$x$10^5$ & 1e-05\% & 1000 & 211.350 ns &  & 5.151 µs \\
$10^5$x$10^5$ & 0.0001\% & 10000 & 380.750 ns &  & 48.725 µs \\
$10^5$x$10^5$ & 0.001\% & 100000 & 775.800 ns &  & 538.387 µs \\
$10^6$x$10^6$ & 1e-06\% & 10000 & 715.550 ns &  & 50.742 µs \\
$10^6$x$10^6$ & 1e-05\% & 100000 & 992.050 ns &  & 541.520 µs \\
$10^6$x$10^6$ & 0.0001\% & 1000000 & 4.347 ms &  & 5.530 ms \\
\bottomrule
\end{tabular}


\pagebreak
\subsection{Transpor}

Transposição é possível considerar que é afetado pela liberação de memoria por causa que,

\begin{enumerate}
    \item Em HashSet possui um tempo aproximadamente constante, por causa que o HashSet possui apenas um bloco gigante de memoria, então é preciso liberar apenas bloco.
    \item Em TableMatrix é afetado porque são $10^3+1$ blocos de memorias de tamanho $10^3$  para serem liberados. 
    \item Em TreeMatrix são diversos blocos de memorias da BTree que precisam ser liberados.
\end{enumerate}
Essa me parece ser a melhor explicação para isso estar acontecendo, tentei implementar os testes sem ser medido a liberação de memoria mas não consegui.

\begin{tabular}{c c c c c c}
\toprule
Tamanho & Ocupação & population & HashMapMatrix & TableMatrix & TreeMatrix \\
\midrule
$10^1$x$10^1$ & 1\% & 1 & 103.250 ns & 1.217 µs & 70.100 ns \\
$10^1$x$10^1$ & 5\% & 5 & 88.450 ns & 1.106 µs & 79.050 ns \\
$10^1$x$10^1$ & 10\% & 10 & 80.550 ns & 1.073 µs & 116.200 ns \\
$10^1$x$10^1$ & 20\% & 20 & 78.700 ns & 1.121 µs & 178.250 ns \\
$10^2$x$10^2$ & 1\% & 100 & 125.200 ns & 37.662 µs & 620.900 ns \\
$10^2$x$10^2$ & 5\% & 500 & 148.900 ns & 31.996 µs & 2.173 µs \\
$10^2$x$10^2$ & 10\% & 1000 & 179.250 ns & 31.546 µs & 4.720 µs \\
$10^2$x$10^2$ & 20\% & 2000 & 227.700 ns & 30.339 µs & 10.673 µs \\
$10^3$x$10^3$ & 1\% & 10000 & 521.250 ns & 9.856 ms & 90.702 µs \\
$10^3$x$10^3$ & 5\% & 50000 & 659.200 ns & 10.350 ms & 227.072 µs \\
$10^3$x$10^3$ & 10\% & 100000 & 777.150 ns & 10.190 ms & 566.442 µs \\
$10^3$x$10^3$ & 20\% & 200000 & 514.800 ns & 10.782 ms & 1.168 ms \\
$10^4$x$10^4$ & 0.0001\% & 100 & 83.500 ns &  & 551.750 ns \\
$10^4$x$10^4$ & 0.001\% & 1000 & 89.250 ns &  & 4.212 µs \\
$10^4$x$10^4$ & 0.01\% & 10000 & 172.600 ns &  & 52.106 µs \\
$10^5$x$10^5$ & 1e-05\% & 1000 & 106.550 ns &  & 4.241 µs \\
$10^5$x$10^5$ & 0.0001\% & 10000 & 176.900 ns &  & 45.456 µs \\
$10^5$x$10^5$ & 0.001\% & 100000 & 538.000 ns &  & 602.392 µs \\
$10^6$x$10^6$ & 1e-06\% & 10000 & 462.400 ns &  & 47.406 µs \\
$10^6$x$10^6$ & 1e-05\% & 100000 & 538.650 ns &  & 560.323 µs \\
$10^6$x$10^6$ & 0.0001\% & 1000000 & 3.309 ms &  & 6.149 ms \\
\bottomrule
\end{tabular}


\pagebreak
\subsection{Multiplicação Escalar}
Nesse caso, claramente é perceptivel, que para TableMatrix, depende puramente do tamanho da matriz e não de quantos elementos possuem na matriz. Onde no caso $10^3\times10^3$ com ocupação menor que 10\% é TableMatrix pior que os outros casos, já com 20\% é melhor que TreeMatrix. 

\begin{tabular}{c c c c c c}
\toprule
Tamanho & Ocupação & population & HashMapMatrix & TableMatrix & TreeMatrix \\
\midrule
$10^1$x$10^1$ & 1\% & 1 & 173.500 ns & 1.273 µs & 193.350 ns \\
$10^1$x$10^1$ & 5\% & 5 & 167.800 ns & 1.111 µs & 202.650 ns \\
$10^1$x$10^1$ & 10\% & 10 & 277.800 ns & 1.125 µs & 257.400 ns \\
$10^1$x$10^1$ & 20\% & 20 & 324.500 ns & 1.140 µs & 601.300 ns \\
$10^2$x$10^2$ & 1\% & 100 & 875.550 ns & 42.736 µs & 2.203 µs \\
$10^2$x$10^2$ & 5\% & 500 & 3.936 µs & 42.345 µs & 11.389 µs \\
$10^2$x$10^2$ & 10\% & 1000 & 7.617 µs & 42.840 µs & 21.201 µs \\
$10^2$x$10^2$ & 20\% & 2000 & 15.423 µs & 39.586 µs & 45.983 µs \\
$10^3$x$10^3$ & 1\% & 10000 & 76.033 µs & 3.749 ms & 213.259 µs \\
$10^3$x$10^3$ & 5\% & 50000 & 469.930 µs & 3.019 ms & 1.216 ms \\
$10^3$x$10^3$ & 10\% & 100000 & 961.508 µs & 3.020 ms & 2.533 ms \\
$10^3$x$10^3$ & 20\% & 200000 & 2.280 ms & 3.017 ms & 5.306 ms \\
$10^4$x$10^4$ & 0.0001\% & 100 & 1.552 µs &  & 2.418 µs \\
$10^4$x$10^4$ & 0.001\% & 1000 & 11.062 µs &  & 24.042 µs \\
$10^4$x$10^4$ & 0.01\% & 10000 & 105.187 µs &  & 217.544 µs \\
$10^5$x$10^5$ & 1e-05\% & 1000 & 7.622 µs &  & 22.811 µs \\
$10^5$x$10^5$ & 0.0001\% & 10000 & 80.119 µs &  & 210.217 µs \\
$10^5$x$10^5$ & 0.001\% & 100000 & 994.864 µs &  & 2.631 ms \\
$10^6$x$10^6$ & 1e-06\% & 10000 & 104.900 µs &  & 210.487 µs \\
$10^6$x$10^6$ & 1e-05\% & 100000 & 1.076 ms &  & 2.838 ms \\
$10^6$x$10^6$ & 0.0001\% & 1000000 & 17.700 ms &  & 26.585 ms \\
\bottomrule
\end{tabular}

\pagebreak
\subsection{ Adição}
Para adição, o TableMatrix se demonstra muito mais eficiente do que as duas outras estruturas, sendo pior apenas nos casos 1\%. Apesar da quantidade de elementos TableMatrix ser maior que as outras estruturas e elementos que precisam ser lidos também. Processadores são extremamente otimizados para leitura e escrita sequencial de elementos, que faz com que TableMatrix se torne mais eficiente.  

\begin{tabular}{c c c c c c}
\toprule
Tamanho & Ocupação & population & HashMapMatrix & TableMatrix & TreeMatrix \\
\midrule
$10^1$x$10^1$ & 1\% & 1 & 644.550 ns & 886.950 ns & 445.750 ns \\
$10^1$x$10^1$ & 5\% & 5 & 848.200 ns & 755.650 ns & 709.400 ns \\
$10^1$x$10^1$ & 10\% & 10 & 1.489 µs & 766.650 ns & 1.699 µs \\
$10^1$x$10^1$ & 20\% & 20 & 3.319 µs & 770.750 ns & 3.854 µs \\
$10^2$x$10^2$ & 1\% & 100 & 11.958 µs & 39.450 µs & 23.220 µs \\
$10^2$x$10^2$ & 5\% & 500 & 80.782 µs & 40.411 µs & 141.043 µs \\
$10^2$x$10^2$ & 10\% & 1000 & 141.338 µs & 42.648 µs & 286.892 µs \\
$10^2$x$10^2$ & 20\% & 2000 & 253.380 µs & 35.940 µs & 511.524 µs \\
$10^3$x$10^3$ & 1\% & 10000 & 1.328 ms & 4.213 ms & 2.333 ms \\
$10^3$x$10^3$ & 5\% & 50000 & 9.592 ms & 3.937 ms & 12.612 ms \\
$10^3$x$10^3$ & 10\% & 100000 & 18.137 ms & 3.811 ms & 26.052 ms \\
$10^3$x$10^3$ & 20\% & 200000 & 44.696 ms & 3.782 ms & 55.186 ms \\
$10^4$x$10^4$ & 0.0001\% & 100 & 12.680 µs &  & 19.994 µs \\
$10^4$x$10^4$ & 0.001\% & 1000 & 140.318 µs &  & 205.606 µs \\
$10^4$x$10^4$ & 0.01\% & 10000 & 1.348 ms &  & 2.252 ms \\
$10^5$x$10^5$ & 1e-05\% & 1000 & 147.205 µs &  & 208.942 µs \\
$10^5$x$10^5$ & 0.0001\% & 10000 & 1.307 ms &  & 2.536 ms \\
$10^5$x$10^5$ & 0.001\% & 100000 & 16.764 ms &  & 26.623 ms \\
$10^6$x$10^6$ & 1e-06\% & 10000 & 2.129 ms &  & 2.311 ms \\
$10^6$x$10^6$ & 1e-05\% & 100000 & 26.515 ms &  & 30.957 ms \\
$10^6$x$10^6$ & 0.0001\% & 1000000 & 392.212 ms &  & 296.280 ms \\
\bottomrule
\end{tabular}


\subsection{Multiplicação}
Mutiplicação aqui aparece como a operação que se tem o maior proveito da esparcidade da matriz,  como no caso $10^3\times10^3$ e 1\%, onde HashMapMatrix é mais de 100x mais performático que  TableMatrix.

Nese caso também HashMapMatrix e TreeMatrix tem valores muito proximos também, apesar de que HashMapMatrix é constantemente mais performatico.

\begin{tabular}{c c c c c c}
\toprule
Tamanho & Ocupação & population & HashMapMatrix & TableMatrix & TreeMatrix \\
\midrule
$10^1$x$10^1$ & 1\% & 1 & 1.774 µs & 1.688 µs & 930.250 ns \\
$10^1$x$10^1$ & 5\% & 5 & 2.377 µs & 1.760 µs & 2.231 µs \\
$10^1$x$10^1$ & 10\% & 10 & 5.196 µs & 1.894 µs & 4.310 µs \\
$10^1$x$10^1$ & 20\% & 20 & 9.396 µs & 1.807 µs & 11.884 µs \\
$10^2$x$10^2$ & 1\% & 100 & 43.326 µs & 1.481 ms & 54.660 µs \\
$10^2$x$10^2$ & 5\% & 500 & 400.137 µs & 1.725 ms & 1.662 ms \\
$10^2$x$10^2$ & 10\% & 1000 & 1.187 ms & 1.594 ms & 2.486 ms \\
$10^2$x$10^2$ & 20\% & 2000 & 4.040 ms & 1.632 ms & 8.941 ms \\
$10^3$x$10^3$ & 1\% & 10000 & 13.442 ms & 1.946 s & 31.565 ms \\
$10^3$x$10^3$ & 5\% & 50000 & 509.380 ms & 1.896 s & 882.779 ms \\
$10^3$x$10^3$ & 10\% & 100000 & 2.175 s & 1.966 s & 3.271 s \\
$10^3$x$10^3$ & 20\% & 200000 & 8.239 s & 2.184 s & 9.949 s \\
$10^4$x$10^4$ & 0.0001\% & 100 & 35.514 µs &  & 53.087 µs \\
$10^4$x$10^4$ & 0.001\% & 1000 & 420.390 µs &  & 555.832 µs \\
$10^4$x$10^4$ & 0.01\% & 10000 & 4.302 ms &  & 7.091 ms \\
$10^5$x$10^5$ & 1e-05\% & 1000 & 355.582 µs &  & 730.562 µs \\
$10^5$x$10^5$ & 0.0001\% & 10000 & 4.135 ms &  & 5.417 ms \\
$10^5$x$10^5$ & 0.001\% & 100000 & 76.117 ms &  & 91.635 ms \\
$10^6$x$10^6$ & 1e-06\% & 10000 & 4.191 ms &  & 5.961 ms \\
$10^6$x$10^6$ & 1e-05\% & 100000 & 85.859 ms &  & 88.136 ms \\
$10^6$x$10^6$ & 0.0001\% & 1000000 & 1.221 s &  & 1.515 s \\
\bottomrule
\end{tabular}



\section{Analise Assintótica}

Foi gerado graficos, onde o eixo X é a quantidade de elementos que cada matriz possui, e o eixo Y é a razão da duração do sample por alguma função assintótica.

Isso é usado por causa de que:
$$\lim_{n\to \infty} \frac{f(x)}{g(x) } \in \mathbb{R}\to f(x) \in \Theta(x)$$
$$\lim_{n\to \infty} \frac{f(x)}{g(x) } = \infty \to f(x) \in \Omega(g(x))$$
$$\lim_{n\to \infty} \frac{f(x)}{g(x) } = 0 \to f(x) \in O(g(x))$$

Portanto, dado isso, podemos definir então por cima, de que caso a razão dos samples estiver tendendo 0 então é $O$, caso esteja crescente então é $\Omega$, caso esteja nem decrescendo indefinidamente nem crescendo indefinidamente então é $\Theta$ 

Nos graficos há tres partes:
\begin{enumerate}
    \item Curva do Supremo: Curva pontos $(k, d)$ que possuem a propriedade $\forall (k', d'), k> k' \to d > d'$, portanto o ponto que é supremo dos valores a partir dele para frente.
    \item Curva do Infimo: Curva dos pontos $(k, d)$ que possuem a propriedade $\forall (k', d'), k> k' \to d < d'$, portanto o ponto que é o infimo dos valores a partir dele para frente.

    \item Pontos Samples coloridos: Os pontos que foram obtidos do experimento, onde são coloridos com base na ocupação. 
    \item Linha media ponderada: Linha horizontal da media ponderada da duração, onde o peso de cada ponto é a população. 
\end{enumerate}

\subsection{TreeMapMatrix}

\subsubsection{set}

Para modificar um valor, tem tempo, que a razão tende a crescer um pouco a mais que $\log k$, porém, extremamente inferior a linear, demonstrando que tem uma aproximação ao estimado que é $\Theta(\log k)$  
\duasfiguras
    {graficos/TreeMatrix/set/1-log_matrix_performance.png}
    {$\log k$}
    {graficos/TreeMatrix/set/2-linear_matrix_performance.png}
    {$k$}
\pagebreak
\subsubsection{get}

Para acesso, as mesmas caracteristicas que set. 
\duasfiguras
    {graficos/TreeMatrix/get/1-log_matrix_performance.png}
    {$\log k$}
    {graficos/TreeMatrix/get/2-linear_matrix_performance.png}
    {$k$}

\subsubsection{Transposição}
O comportamento da transposição tende a parecer linear ao crescer, mas demonstra uma taxa de crescimento.
\duasfiguras
    {graficos/TreeMatrix/transpose/0-constant_matrix_performance.png}
    {$1$}
    {graficos/TreeMatrix/transpose/1-log_matrix_performance.png}
    {$k$}

    
\subsubsection{Multiplicação Escalar}
O grafico se demonstra crescente em relação a $\log k$ e decrescente em relação a $k$, portanto é uma complexidade pior que logaritmica, mas melhor que linear 
\duasfiguras
    {graficos/TreeMatrix/muls/1-log_matrix_performance.png}
    {$\log k$}
    {graficos/TreeMatrix/muls/2-linear_matrix_performance.png}
    {$k$}
\pagebreak
\subsubsection{Adição}
O comportamento para soma, tem uma tendencia extremamente alta para convergir na razão pelo linear, mostrando um comportamento claramente na parte testada $\Theta(k)$, melhor do que o calculado.
\duasfiguras
    {graficos/TreeMatrix/add/2-linear_matrix_performance.png}
    {$k$}
    {graficos/TreeMatrix/add/3-nlog_matrix_performance.png}
    {$k\log k$}
    
\subsubsection{Multiplicação Matrizes}

Para multiplicação de matrizes, o algoritmo demonstra ter um comportamento próximo de $k \log k \sqrt k$  porém com uma tendencia de crescimento, porém com uma tendencia decrescente em relação a $k^2$.

\duasfiguras
    {graficos/TreeMatrix/mul/5-nlogsqrt_matrix_performance.png}
    {$k \log k \sqrt k$}
    {graficos/TreeMatrix/mul/6-quadratic_matrix_performance.png}
    {$k^2$}



\pagebreak

\subsection{HashMapMatrix}

\subsubsection{set}

Para o set, apesar do esperado ter sido estimado como constante, as métricas tenderam  para estar se aproximando em relação a $\log k$, e ser muito menor que linear.
\duasfiguras
    {graficos/HashMapMatrix/set/1-log_matrix_performance.png}
    {$\log k$}
    {graficos/HashMapMatrix/set/2-linear_matrix_performance.png}
    {$k$}
\subsubsection{get}
Para acesso, as mesmo comportamento que set. 
\duasfiguras
    {graficos/HashMapMatrix/get/1-log_matrix_performance.png}
    {$\log k$}
    {graficos/HashMapMatrix/get/2-linear_matrix_performance.png}
    {$k$}
\pagebreak
\subsubsection{Transposição}
O comportamento da transposição está próximo do comportamento do get e do set, mostrando que provavelmente deve ser um problema em relação ao experimento, o fato de estar mais próximo de log do que constante.
\duasfiguras
    {graficos/HashMapMatrix/transpose/1-log_matrix_performance.png}
    {$1$}
    {graficos/HashMapMatrix/transpose/2-linear_matrix_performance.png}
    {$k$}

    
\subsubsection{Multiplicação Escalar}
Na multiplicação de escalar é possivel perceber que tem uma performance suavemente melhor que $k \log k$ e suavemente pior que $k$. Mas além disso um detalhe importante é perceber o comportamente ocilatorio do valor, que se dá ao quanto da tabela é ocupada, onde quando temos uma tabela com a capacidade proxima do tamanho, temos uma melhor performance, e como a ocupação da tabela vai variando, então a performance varia, mas se mantêm aproximadamente a razão.
\duasfiguras
    {graficos/HashMapMatrix/muls/2-linear_matrix_performance.png}
    {$k$}
    {graficos/HashMapMatrix/muls/3-nlog_matrix_performance.png}
    {$k\log k$}
\pagebreak
\subsubsection{Adição}
Tem um comportamento um pouco pior que linear, e um pouco melhor que log-linear, que provavelmente é causado pelas colisões de hash.
\duasfiguras
    {graficos/HashMapMatrix/add/2-linear_matrix_performance.png}
    {$k$}
    {graficos/HashMapMatrix/add/3-nlog_matrix_performance.png}
    {$k\log k$}
    
\subsubsection{Multiplicação Matrizes}

Para multiplicação de matrizes, o algoritmo demonstra ter um comportamento próximo de $k \log k \sqrt k$  porém com uma tendencia de crescimento, porém com uma tendencia decrescente em relação a $k^2$, então como estimado é melhor que $k^2$. 

\duasfiguras
    {graficos/HashMapMatrix/mul/5-nlogsqrt_matrix_performance.png}
    {$k \log k \sqrt k$}
    {graficos/HashMapMatrix/mul/6-quadratic_matrix_performance.png}
    {$k^2$}




\pagebreak  

\subsection{TableMatrix}
Para TableMatrix, a analise assintótica em relação ao quantidade de elementos não nulo, não faz sentido, porque não existe um limite superior de quão ruim um caso pode ser, onde o caso sempre cresce indefinidamente junto ao tamanho da matriz e não em relação a quantidade de elementos não nulos. 

Isso é perceptivel nos graficos de que há uma clara separação na maior parte deles do tempo em relação a cada taxa de ocupação (indicada pelas cores dos pontos).

\begin{figure}[h]
    \centering
    \begin{minipage}{0.45\textwidth}
        \centering
        \includegraphics[width=\linewidth]{graficos/TableMatrix/set/0-constant_matrix_performance.png}
        \caption{Set}
    \end{minipage}
    \hfill
    \begin{minipage}{0.45\textwidth}
        \centering
        \includegraphics[width=\linewidth]{graficos/TableMatrix/get/0-constant_matrix_performance.png}
        \caption{Get}
    \hfill
    \end{minipage}
        \begin{minipage}{0.45\textwidth}
        \centering
        \includegraphics[width=\linewidth]{graficos/TableMatrix/transpose/0-constant_matrix_performance.png}
        \caption{Transpor}
    \end{minipage}
    \hfill
    \begin{minipage}{0.45\textwidth}
        \centering
        \includegraphics[width=\linewidth]{graficos/TableMatrix/muls/0-constant_matrix_performance.png}
        \caption{Multiplicação escalar}
    \end{minipage}
    \hfill
    \begin{minipage}{0.45\textwidth}
        \centering
        \includegraphics[width=\linewidth]{graficos/TableMatrix/add/0-constant_matrix_performance.png}
        \caption{Adição}
    \end{minipage}
    \hfill
    \begin{minipage}{0.45\textwidth}
        \centering
        \includegraphics[width=\linewidth]{graficos/TableMatrix/mul/0-constant_matrix_performance.png}
        \caption{Multiplicação Matriz}
    \end{minipage}
\end{figure}

\pagebreak

Pegando a multiplicação de matriz em destaque é perceptivel, que a linha do supremo segue exatamente uma linha de cores iguais, que é exatamente os pontos do $1\%$ de ocupação, onde é 1\% de matrizes muito maiores que a população, então se diverge das outras.


\begin{figure}[h]
    \centering
    \begin{minipage}{0.6\textwidth}
        \centering
        \includegraphics[width=\linewidth]{graficos/TableMatrix/mul/0-constant_matrix_performance.png}
        \caption{Multiplicação Matriz}
    \end{minipage}
\end{figure}
Mas caso, plotamos os valores, mas ao invés de colocarmos no eixo x a quantidade de elementos não nulos, colocarmos a quantidade de elementos no total, é perceptível que os dados são muito mais correlacionados. 
\begin{figure}[h]
    \centering
    \begin{minipage}{0.6\textwidth}
        \centering
        \includegraphics[width=\linewidth]{graficos/TableMatrix/mul/size_matrix_performance.png}
        \caption{Multiplicação Matriz}
    \end{minipage}
\end{figure}