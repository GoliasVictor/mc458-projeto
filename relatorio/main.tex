\documentclass{article}
\usepackage{graphicx} % Required for inserting images
\usepackage[portuguese]{babel}
\usepackage{amsthm}
\usepackage{amsmath}
\usepackage{amssymb}
\usepackage{booktabs}
\newcommand{\mdc}{\text{mdc}}
\usepackage[a4paper, margin=2cm]{geometry}
\newcommand{\N}{\mathbb{N}}
\newcommand{\R}{\mathbb{R}}
\theoremstyle{definition}
\newtheorem{pergunta}{Pergunta}
\newtheorem{subperg}{Parte}[pergunta]
\newtheorem{lema}{Lema}[pergunta]
\newtheorem{definicao}{Definição}
\usepackage{mathtools}
\DeclarePairedDelimiter\ceil{\lceil}{\rceil}
\DeclarePairedDelimiter\floor{\lfloor}{\rfloor}
\usepackage{algorithm}
\usepackage{algpseudocode}
\newtheorem{prop}{Propriedade}[definicao]
\title{MC458 - Lista 1}
\author{José Victor Santana Barbosa - 245511}
\date{21 de Novembro 2025}
\title{MapMatrix}
 
\newcommand{\duasfiguras}[5][0.45]{%
    \begin{figure}[h]
        \centering
        \begin{minipage}{#1\textwidth}
            \centering
            \includegraphics[width=\linewidth]{#2}
            \caption{#3}
        \end{minipage}
        \hfill
        \begin{minipage}{#1\textwidth}
            \centering
            \includegraphics[width=\linewidth]{#4}
            \caption{#5}
        \end{minipage}
    \end{figure}
}

\begin{document} 
\maketitle
\tableofcontents
\section{Resumo}
O seguinte trabalho, apresenta sobre a implementação de uma estrutura de dados chamada MapMatrix, que representa matrizes esparsas utilizando mapas associativos. A estrutura é genérica e pode ser utilizada com diferentes tipos de armazenamento subjacente, desde possuam interface com certas operações básicas. A seguir, serão detalhadas as principais funcionalidades e características da estrutura MapMatrix.

\section{Introdução}

MapMatrix parte da ideia, de ao invés de representar representar uma matriz como um vetor bidimensional, onde a posição na memoria física do computador tem relação direta com a posição logica da matriz, utilizar uma estrutura de dados chamada mapa associativo (ou dicionario), onde cada elemento da matriz é armazenado como um par chave-valor, onde a chave é a posição do elemento na matriz (representada como uma tupla de índices) e o valor é o valor do elemento na matriz.

\subsection{Iteradores}
Um recurso que é vastamente utilizado neste projeto são iteradores\footnote{https://doc.rust-lang.org/std/iter/trait.Iterator.html}, que servem para percorrer os valores de alguma estrutura de dados. Na interface de um Iterador é obrigatória a existência de apenas uma única operação, que é a operação de \textit{next()}. Essa operação, que caso haja algum valor ainda no iterador, retorna o próximo valor; e, caso não haja mais nenhum valor disponível, retorna vazio.

\subsection{Mapas}
Um mapa associativo é uma estrutura de dados que armazena pares chave-valor, permitindo a associação de um valor a uma chave única. Para um mapa associativo ser usado em MapMatrix é necessário que ele possua as seguintes operações:
\begin{itemize}
	\item \textbf{new()}: Cria um novo mapa vazio.
	\item \textbf{get(key)}: Retorna o valor associado a chave key, ou um valor padrão caso a chave não exista no mapa.
	\item \textbf{set\_or\_insert(key, value)}: Insere o par chave-valor (key, value) no mapa. Se a chave ja existir, atualiza o valor associado a chave.
	\item \textbf{remove(key)}: Remove o par chave-valor associado a chave key do mapa.
	\item \textbf{iter() / iter_mut()}: Retorna um iterador sobre os pares chave-valor do mapa.
	\item \textbf{from\_iter(iter)}: Cria um mapa a partir de um iterador que produz pares chave-valor.
    \item \textbf{clone()}: Faz uma copia completa de toda a estrutura.
\end{itemize}

\section{MapMatrix}

A estrutura MapMatrix para ser definida e usada é preciso de dois tipos de mapas auxiliares, um Mapa que armazena os valores da matriz, sendo a chave a posição do elemento e o valor o valor do elemento, e o outro mapa é usando na operação de multiplicação de matrizes, onde é um mapa de índices (colunas ou linhas) para lista de valores.
Antes de detalhar as operações da estrutura MapMatrix, primeiro vamos falar sobre um tipo especial de mapa associativo que é usado na implementação, chamado TransposableMap.
\subsection{TransposableMap}
Para que execução eficiente de transposição de matrizes, foi criado uma estrutura chamada TransposableMap, que ela basicamente encapsula um mapa associativo com chaves sendo pares de índices e adiciona a funcionalidade de transposição.

Além das funcionalidades básicas de um Mapa Associativo, TransposableMap adiciona a funcionalidade de 
transpor, que faz inversão de um uma flag interna indicando que a matriz foi transposta.

A partir da flag interna, todas as operações sobre o mapa, é verificado se a flag está ativa, caso sim, inverte os índices das chaves antes de realizar a operação e ao retornar os valores. Caso não, executa as operações normalmente. 

A partir dessa estrutura, é possível implementar a operação de transposição de uma matriz em tempo constante, apenas invertendo a flag interna, sem a necessidade de iterar sobre todos os elementos da matriz e trocar suas posições. E sem ser necessário dentro da estrutura MapMatrix, adicionar lógica extra para lidar com a transposição. Já que a o único ponto de entrada e saída de dados da matriz é o mapa encapsulado.
\subsection{Operações de MapMatrix}
A seguir, são detalhadas as principais operações implementadas na estrutura MapMatrix:
Essa sessão as equações são um pouco grandes, mas isso é por causa que é feito de forma genérica, então depende de como o Mapa é implementado, em sessões posteriores as complexidades são descritas de forma mais simples.
Vamos usar a seguinte notação 
\begin{enumerate}
    \item $k, k_A, k_B$: Numero de elementos não nulos na matriz.
    \item $|A_C|$, $|B_L|$: Numero de colunas em A, e o numero de linhsa em B.
    \item $|A_{c=i}|$: Numero de elementos em uma coluna $i$ na matriz A
    \item $|B_{l=i}|$: Numero de elementos em uma linhha $i$ na matriz B 
    \item $M_{nome}(k)$ Tempo para execução de uma operação no Mapa com o nome especificado
    
    \item $LM_{nome}(k)$ Tempo para execução de uma operação no Mapa de Vetor com o nome especificado
    \item $U(k)$ Tempo para execução de uma operação no Mapa com o nome especificado
    
\end{enumerate}

nota: para complexidade de memoria será considerado o custo de chamada das operações em Map é desprezível.  



Será usado a seguinte
\subsubsection{Acessar elemento}
O acesso de elemento é simples, é basicamente uma leitura do mapa associativo encapsulado, passando a posição do elemento como chave. Caso o elemento não exista na matriz, é retornado 0.



\begin{enumerate}
    \item \textbf{Complexidade de tempo:} $M_{get}(k) $,
    \item \textbf{Complexidade de espaço:} $\Theta(1)$,
\end{enumerate}
\subsubsection{Transpor matriz}
A operação de transpor matriz, apenas inverte o tamanho da linha/coluna e troca a flag de transposição, então tem custo constante.

\begin{enumerate}
    \item \textbf{Complexidade de tempo:} $\Theta(1)$,
    \item \textbf{Complexidade de espaço:} $\Theta(1)$,
\end{enumerate}

\subsubsection{Multiplicação por escalar}
A multiplicação por escalar é feita copiando o mapa, e então usando um iterador mutável, para cada valor multiplicação pelo escalar. E então retorna esse novo valor
\begin{enumerate}
    \item \textbf{Complexidade de tempo:}  
    $$M_{clone} + M_{iter}(k) $$,
    A complexidade de tempo é o custo de clonar  e de iterar mutiplicando cada valor da matriz
    \item \textbf{Complexidade de espaço:} 
    $$U(k)$$, 
    O custo de espaço é o custo da nova matriz criada.
\end{enumerate}


\subsubsection{Soma de matrizes}
A soma de matrizes primeiro clona $A$ criando a matriz $C$, então, itera por $B$ e obtêm o valor de $A$ na posição soma e guarda em $C$.
\begin{enumerate}
    \item \textbf{Complexidade de tempo:}  
    $$M_{clone}(k_A) + M_{iter}(k_B) +  k_B(M_{set}(k_C) + M_{get}(k_A))$$
    A complexidade de tempo é o custo de copiar $A$, e o custo para cada valor de $B$, ler de $A$ e escrever em $C$  
    \item \textbf{Complexidade de espaço:} 
    $$U(k_C) + U_{set}(k) + U_{set}(k)$$, 
    O custo de espaço é o custo da nova matriz criada, mas o custo de espaço para executar a operação de set e get.
\end{enumerate}

\subsubsection{Mutiplicação de matrizes}
Para a mutiplicação de matriz temos como base a seguinte formula:
$C_{i, j} = \sum_{k=1}^n A_{i, k} B_{k, j}$
E como é perceptível, para mutiplicação de matriz, os elementos na numa coluna $k$ de $A$ só será mutiplicado com elementos de uma linha $k$ em $B$, significa.

Portanto, o que fazemos, é então, é primeiro separar os elementos de $A$ por linha, e o de $B$ por coluna, e então multiplicamos cada linha pela respectiva coluna e somamos em $C$.

Que o algoritmo possui duas partes principais:

\textbf{Separação:} É criado um mapa das colunas de $A$ e um das linhas de $B$, e então iteramos em $A$ adicionando nas sua respectiva coluna, e iteramos em $B$ os adicionando nas suas respectivas $linhas$.

\textbf{Multiplicação:} Então, para cada linha de $A$ é feito a leitura da coluna correspondente em $B$, e então iterado em cada um dos elementos na linha e coluna, e multiplicados e somados a $C$
\begin{enumerate}
    \item \textbf{Complexidade de tempo:}  
    $|A_C|$ representa a quantidade de colunas significativas em  $A$,
    $|B_L|$ representa a quantidade de colunas significativas em $B$
    \begin{enumerate}
        \item \textbf{Separação:} 
            $$M_{iter}(k_A)+ M_{iter}(k_B) + k_A LM_{add}(|A_C|) +  k_B LM_{add}(|B_L|)$$

        A complexidade de tempo é o custo de iterar em cada uma das matrizes, mais o custo de para cada um dos valores adicionar um elemento a mais, onde o tamanho de cada um dos mapa de vetores, é a quantidade de colunas relevantes em $A$ e a quantidade de colunas relevantes em $B$
        \item \textbf{Multiplicação:}
            Na multiplicação esse é o custo para iterar em cada coluna de $A$ e ler uma linha em $B$
            $$LM_{iter}(|A_C|) + |A_C| LM_{get}(|B_L|)$$
            
            Como guardamos cada elemento de cada linha/coluna num vetor, sabemos que tem um custo linear iterar neles, e o custo da nossa multiplicação se dá pela quantidades de elementos na coluna  ($A_{c=n}$) e a quantidade de elementos da linha ($B_{l=n}$).  Porque iremos fazer cada uma das combinações de elementos na linha e coluna, e para cada combinação iremos ler de ser e atribuir um novo valor, portanto:
            $$\sum _{i= 1}^n  |A_{c=i}| |B_{l=i}| (M_{get}(k_C) +  M_{set}(k_C))$$.
            
    \end{enumerate}
    
    \item \textbf{Complexidade de espaço:} 
    $$U(k_C) +U(k_A) + U(k_B)$$, 
    O custo de espaço é o custo da nova matriz criada, mais o custo de manter o mapa auxiliar de colunas e linhas. 
\end{enumerate}


\section{Implementações}
\subsection{HashMapMatrix}

Essa é a primeira estrutura de dados solicitada, que é a execução de uma MapMatrix usando um HashMap como Mapa.

Essa estrutura usa uma Tabela  Hash, especificamente a Tabela Hash da biblioteca padrão do Rust\footnote{https://doc.rust-lang.org/std/collections/struct.HashMap.html}, que é um porte da tabela SwissTable\footnote{https://abseil.io/blog/20180927-swisstables} desenvolvida pelo Google.

Para a maior parte das operações que usamos do HashMap a complexidade está definida na propria documentação \footnote{https://doc.rust-lang.org/std/collections/\#cost-of-collection-operations}.

Na documentação explicita que as operações de \textit{get}, \textit{insert} e \textit{remove}, possuem complexidade esperada $O(1)$. Que é o caso onde não ocorrem muitas colisões de hash, que é o caso que vamos estar considerando aqui.

Em relação a iterar em todos valores de uma hash table, é especificado na documentação\footnote{https://doc.rust-lang.org/std/collections/struct.HashMap.html\#performance-5}, que na iteração é visitado também os buckets vazios, então é $O(capacidade)$ e não $O(tamanho)$ de elementos no hashset, porém como é uma implementação de hashtable, e a capacidade está crescendo junto com a adição de elementos, e não definida manualmente, é razoavel considerar que a implementação usa mecanismos de quando ultrapassar o fator de carga, a capacidade aumentar por um fator de crescimento, e quando é removido uma certa quantidade de elementos é também feito reconstrução da tabela liberando memoria, portanto vou considerar que é razoável especificar que $capacidade \in \Theta(tamanho)$. (na parte de metricas será visto na pratica que é  isso que acontece)

Em relação a operação de clone, o possui basicamente a mesma questão da iteração, que basicamente copia toda a tabela e isso é relativo a capacidade da tabela.

Portanto temos que 
\begin{itemize}
	\item \textbf{new()}: $\Theta(1)\sim$
	\item \textbf{get(key)}: $\Theta(1)\sim$
	\item \textbf{set\_or\_insert(key, value)}: $\Theta(1)*\sim$
	\item \textbf{remove(key)}: $\Theta(1)*\sim$
	\item \textbf{iterar todos elementos}: $\Theta(k)$ 
	\item \textbf{from\_iter(iter)}: $\Theta(k)\sim$
	\item \textbf{clone()}: $\Theta(k)$
\end{itemize}
$\sim$ significa complexidade esperada, * significa complexidade amortizada.

Portando, agora sobre a complexidade de cada uma das operações

E para operação de adição a um vetor no caso de mapa de vetores, como é apenas um get, então também é constante
\subsubsection{Acessar elemento}
\begin{enumerate}
    \item \textbf{Complexidade de tempo  esperado:} $\Theta(1) $,
    \item \textbf{Complexidade de espaço:} $\Theta(1) $,
\end{enumerate}

\subsubsection{Transpor matriz}
\begin{enumerate}
    \item \textbf{Complexidade de tempo  esperado:} $\Theta(1)$,
    \item \textbf{Complexidade de espaço:} $\Theta(1)$,
\end{enumerate}

\subsubsection{Multiplicação por escalar}
\begin{enumerate}
    \item \textbf{Complexidade de tempo  esperado:}  
    $$\Theta(k) + \Theta(k)$$
    $$\Theta(k)$$
    
    \item \textbf{Complexidade de espaço:} $\Theta(k)$,

\end{enumerate}


\subsubsection{Soma de matrizes} 
\begin{enumerate}
    \item \textbf{Complexidade de tempo  esperado:}  
    $$M_{clone}(k_A) + M_{iter}(k_B) +  k_B(M_{set}(k_C) + M_{get}(k_A))$$
    $$\Theta(k_A) + \Theta(k_B) +    k_B(\Theta(1) + \Theta(1))$$
    $$\Theta(k_A  + k_B)$$
    \item \textbf{Complexidade de espaço:} 
    $$\Theta(k_C)$$, 
\end{enumerate}

\subsubsection{Mutiplicação de matrizes}

\begin{enumerate}
    \item \textbf{Complexidade de tempo esperado:}  
    $|A_C|$ representa a quantidade de colunas significativas em $A$,
    $|B_L|$ representa a quantidade de colunas significativas em $B$
    \begin{enumerate}
        \item \textbf{Separação:} 
        $$M_{iter}(k_A)+ M_{iter}(k_B) + k_A LM_{add}(|A_C|) +  k_B LM_{add}(|B_L|)$$
        $$\Theta(k_A) + \Theta(k_B) + k_A \Theta(1)+  k_B \Theta(1)$$
        $$\Theta(k_A + k_B)$$ 
        \item \textbf{Multiplicação:}
            Na multiplicação esse é o custo para iterar em cada coluna de $A$ e ler uma linha em $B$
            $$LM_{iter}(|A_C|) + |A_C| LM_{get}(|B_L|)$$
            $$\Theta(|A_C|) + |A_C|\Theta(1)$$
            $$\Theta(|A_C|)$$
            Custo para mutiplicar cada linha por cada coluna
            $$\sum _{i= 1}^n  |A_{c=i}| |B_{l=i}| (M_{get}(k_C) +  M_{set}(k_C))$$
            $$\sum _{i= 1}^n  |A_{c=i}| |B_{l=i}| (\Theta(1) +  \Theta(1))$$
            $$\Theta(1)\sum _{i= 1}^n  |A_{c=i}| |B_{l=i}|$$
            Para calcularmos a esperança, vamos usar as seguintes duas variaveis indicadores:
            $$X_{a, i} = \begin{cases}
                1 &, \text{se coluna de $a$ é $i$}
                \\0 &, \text{se não}
            \end{cases}$$
            
            $$Y_{b, i} = \begin{cases}
                1 &, \text{se linha de $b$ é $i$}
                \\0 &, \text{se não}
            \end{cases}$$
            No qual, como está uniformemente distribuído, a chance de uma variável  estar em uma linha/coluna de tamanho $n$ é 
            $$E[X_{a, i}] = E[Y_{b, i}] = \frac{1}{n}$$
            Então definimos 
            $$|A_{c=i}| = \sum_{a\in A} X_{a, i}$$
            $$|B_{l=i}| = \sum_{b\in B} Y_{b, i} $$
            Logo
            $$E[|A_{c=n}|] = \sum_{a\in A} E[X_{a, i}] = \frac{1}{n}\sum_{a\in A} 1 = \frac{k_A}{n}$$
            $$E[|B_{l=n}|] = \sum_{b\in B} E[Y_{b, i}] = \frac{1}{n}\sum_{b\in B} 1 = \frac{k_B}{n} $$
            
            Então reescrevemos a complexiade esperada dessa parte da seguinte forma:
            $$\text E\left[\sum _{i= 1}^n  |A_{c=i}| |B_{l=i}|\right] = \sum _{i= 1}^n  \text E\left[|A_{c=i}|\right] \text E\left[|B_{l=i}|\right]  $$
            Logo 
            $$\sum _{i= 1}^n  \text E\left[|A_{c=i}|\right] \text E\left[|B_{l=i}|\right] = \sum _{i= 1}^n \frac{k_A}{n}\frac{k_B}{n} = n\frac{k_A}{n}\frac{k_B}{n} =  \frac{k_Ak_B}{n} $$            
            Logo então a complexidade esperada de mutilicar as linhas pelas colunas é:
            $$\Theta(|A_C|+ \frac{k_Ak_B}{n})$$
            
    \end{enumerate}
    Juntando então cada parte temos que a complexidade total da mutiplicação de matriz é
    $$\Theta\left( k_A+ k_B+ |A_C|+ \frac{k_Ak_B}{n}\right)$$
    Podemos considerar definir a media de elementos por linha em $B$, 
    $$d_B = \frac{k_B}{n}$$
    E também ignorar $|A_C|$ porque $k_A > |A_C|$
    Então temos:
    $$\Theta\left(k_B+ k_Ad_B\right)$$
    E como $d_B < k_B$
    $$k_B+ k_Ad_B \in O(k_Ak_B)$$
    
    
    \item \textbf{Complexidade de espaço:} 
    $$U(k_C) +U(k_A) + U(k_B)$$, 
    O custo de espaço é o custo da nova matriz criada, mais o custo de manter o mapa auxiliar de colunas e linhas. 
\end{enumerate}

\subsection{TreeMapMatrix}

A segunda estrutura de dados, com requisitos garantidos, foi implementada usando o MapMatrix com o Mapa associativo usado para guardar os dados sendo uma BTree.

Como o HashMap, foi implementado usando a implementação padrão de BTree da biblioteca padrão no Rust\footnote{\label{btree}https://doc.rust-lang.org/std/collections/struct.BTreeMap.html}. E possui complexidade também documentada para cada operação \footnote{https://doc.rust-lang.org/std/collections/\#cost-of-collection-operations} exceto a de iterar sobre todos valores.

Mas para a operação de iteração na pagina sobre BTree\footnote{\label{btree}}, é descrito que a operação de produzir o próprio valor no iterador tem custo $\Theta(\log n)$ no pior caso, mas amortizado possui tempo constante por item $\Theta(1)$, portanto, para ler todos valores, temos tempo linear.

Para a operação de clone é possivel observar o codigo fonte\footnote{https://doc.rust-lang.org/stable/src/alloc/collections/btree/map.rs.html#224} e ver que é feito clone recursivamente da arvore, portanto trivialmente é linear em relação ao tamanho da arvore. 


Portanto temos que 
\begin{itemize}
	\item \textbf{new()}: $\Theta(1)$
	\item \textbf{get(key)}: $\Theta(\log k)$
	\item \textbf{set\_or\_insert(key, value)}: $\Theta(\log k)$
	\item \textbf{remove(key)}: $\Theta(\log k)$
	\item \textbf{iterar todos elementos}: $\Theta(k)$ 
	\item \textbf{from\_iter(iter)}: $\Theta(k)$
    \item \textbf{clone()}: $\Theta(k)$
\end{itemize}

Portando, agora sobre a complexidade de cada uma das operações

E para operação de adição a um vetor no caso de mapa de vetores, como é apenas um get, então também é constante
\subsubsection{Acessar elemento}
\begin{enumerate}
    \item \textbf{Complexidade de tempo:} $\Theta(\log k) $,
    \item \textbf{Complexidade de espaço:} $\Theta(1) $,
\end{enumerate}

\subsubsection{Transpor matriz}
\begin{enumerate}
    \item \textbf{Complexidade de tempo:} $\Theta(\log k)$,
    \item \textbf{Complexidade de espaço:} $\Theta(1)$,
\end{enumerate}

\subsubsection{Multiplicação por escalar}
\begin{enumerate}
    \item \textbf{Complexidade de tempo:}  
    $$\Theta(k) + \Theta(k)$$
    $$\Theta(k)$$
    
    \item \textbf{Complexidade de espaço:} $\Theta(k)$,

\end{enumerate}


\subsubsection{Soma de matrizes} 
\begin{enumerate}
    \item \textbf{Complexidade de tempo:}  
    $$M_{clone}(k_A) + M_{iter}(k_B) +  k_B(M_{set}(k_C) + M_{get}(k_A))$$
    $$\Theta(k_A) + \Theta(k_B) +  k_B(\Theta(\log k_C) + \Theta(\log k_A))$$
    $$\Theta(k_A + k_B +  k_B(\log k_C +\log k_A))$$
    Sabendo que  $k_B \in O(k_B \log k_C)$
    A complexidade final é a seguinte:
    $$\Theta(k_A + k_B \log k_C)$$

    E como $k_A \in O(k_A \log k_C)$
    Então 
    $$k_A + k_B \log k_C \in  O((k_A +k_B)k_C)$$
    \item \textbf{Complexidade de espaço:} 
    $$\Theta(k_C)$$, 
\end{enumerate}

\subsubsection{Mutiplicação de matrizes}

\begin{enumerate}
    \item \textbf{Complexidade de tempo:}  
    $|A_C|$ representa a quantidade de colunas significativas em $A$,
    $|B_L|$ representa a quantidade de colunas significativas em $B$
    \begin{enumerate}
        \item \textbf{Separação:} 
        $$M_{iter}(k_A)+ M_{iter}(k_B) + k_A LM_{add}(|A_C|) +  k_B LM_{add}(|B_L|)$$
        $$\Theta(k_A) + \Theta(k_B) + k_A \Theta(\log |A_C|)+  k_B \Theta(\log |B_L|)$$
        $$\Theta(k_A\log |A_C|  + k_B\log |B_L|)$$ 
        \item \textbf{Multiplicação:}
            Na multiplicação esse é o custo para iterar em cada coluna de $A$ e ler uma linha em $B$
            $$LM_{iter}(|A_C|) + |A_C| LM_{get}(|B_L|)$$
            $$\Theta(|A_C|) + |A_C|\Theta(\log |B_L|)$$
            $$\Theta(|A_C|\log |B_L|)$$
            Custo para mutiplicar cada linha por cada coluna
            $$\sum _{i= 1}^n  |A_{c=i}| |B_{l=i}| (M_{get}(k_C) +  M_{set}(k_C))$$
            $$\sum _{i= 1}^n  |A_{c=i}| |B_{l=i}| (\Theta(\log k_C) +  \Theta(\log k_C))$$
            $$\Theta(\log k_C)\cdot\sum _{i= 1}^n  |A_{c=i}| |B_{l=i}|$$

            Onde o nosso pior caso é quanto todos elementos de $|A|$ estão em uma única coluna, e todos elementos de $B$ estão em uma única linha. Onde o somatorio resulta em $k_A k_B$
            
            Podemos provar isso por absurdo, considerando que existe alguma distribuição $|A_{c=i}'|$ de elementos em $A$ e $|B_{c=i}'|$ de elementos em $B$, que o pior caso.
            Poranto para essa sequencia ser pior que todos em uma unica linha/coluna, significa então:
                $$\sum _{i= 1}^n  |A_{c=i}'| |B_{l=i}'| > k_A k_B$$
            E como 
            $$k_A = \sum_{i=1}^n |A_{c=i}'|  $$
            $$k_B = \sum_{i=1}^n |B_{l=i}'| $$
            Logo isso significa que:
            \begin{align*}
                 \sum _{i= 1}^n  |A_{c=i}'| |B_{l=i}'| &> k_A k_B
                 \\ &> \left( \sum_{i=1}^n |A_{c=i}'|\right)  \left( \sum_{i=1}^n |B_{l=i}'| \right) 
                 \\ &>  \sum_{i=1}^n \sum_{i=1}^n  |A_{c=i}'| |B_{l=j}'| 
                 \\ &> \sum _{i= 1}^n  |A_{c=i}'| |B_{l=i}'|  +  \sum_{i=1}^n \sum_{j=1, j\neq i}^n  |A_{c=i}'| |B_{l=j}'|  
            \end{align*}

            Portanto:
            $$0> \sum_{i=1}^n \sum_{j=1, j\neq i}^n  |A_{c=i}'| |B_{l=j}'|  $$
            Que é impossível porque uma soma de tamanhos de conjunto deve ser no mínimo positiva. Portanto é uma contradição e não existe uma distribuição que o resultado seja maior.

            Logo chegamos que a complexidade da mutiplicação das colunas pelas linhas no pior caso é quando quando todos elementos de B estão numa linha apenas e todos de A estão numa coluna só logo:
            $$\Theta\left(|A_C|\log |B_L| + k_Ak_B\log k_C\right)$$
            $$\Theta\left(1\log 1 + k_Ak_B\log k_C\right)$$
            $$\Theta\left(k_Ak_B\log k_C\right)$$
            
    \end{enumerate}
    Juntando então cada parte temos que a complexidade total da mutiplicação de matriz é
    $$\Theta\left( k_A\log |A_C|  + k_B\log |B_L| + k_A k_B\log k_C\right)$$
    E como no pior caso tudo estar numa coluna/linha apenas:
    $$\Theta\left( k_A\log 1  + k_B\log 1 + k_A k_B\log k_C\right)$$

    Chegamos a seguinte complexidade de tempo final:
    $$\Theta\left(k_A k_B\log k_C\right)$$
    
    
    
    \item \textbf{Complexidade de espaço:} 
    $$U(k_C) +U(k_A) + U(k_B)$$, 
    O custo de espaço é o custo da nova matriz criada, mais o custo de manter o mapa auxiliar de colunas e linhas. 
\end{enumerate}

\section{Analise}
Agora irei descrever a analise experimental, na qual eu fiz de duas formas.
\begin{enumerate}
    \item Grafica: Fiz experimentos com matrizes quadradas de lado no intervalo $[10^{1}, 10^{3}]$, no qual é feito samples igualmente espaçados. E assim gerei graficos a partir disso.
    \item Tabular: Fiz experimentos com matrizes quadradas de lado no intervalo $[10^{1}, 10^{6}]$, para cada potencia de 10, é estimado o tempo medio  
\end{enumerate}

Estou me referindo por TableMatrix, como a implementação que faz operações com uma representação de linhas e vetores na memoria. 

\section{Comparação}

Para todas as operações que eram para ter uma estimativa de tempo constante, apesar de ser estimativamente constante, há um crescimento mesmo em TableMatrix que faz acesso direto a memoria. 
Suponho que o problema disso, seja alocamento e desalocamento de memoria nos brenchmarks que deve estar acontecendo.  Por causa que as matrizes são criadas e logo após não usadas nunca mais, então logo após de ter sido aplicado a operação o programa já libera a memoria. 

Para TableMatrix, só foi possivel fazer samples até $10^3$, por causa  que além disso, a há estouro de memoria.

\subsection{Get}
Na leitura de valores, é perceptível que para hashmap é perceptivel que é o que tem a melhor performance que todos os outros, tendo tempo aproximadamente constante em nanosegundos para todos valores.

Provavelmente em TableMatrix há uma perda de desempenho de memoria da memoria com o crescimento da matriz por causa de paginação da memoria.

\begin{tabular}{c c c c c c}
\toprule
Tamanho & Ocupação & population & HashMapMatrix & TableMatrix & TreeMatrix \\
\midrule
$10^1$x$10^1$ & 1\% & 1 & 161.600 ns & 314.000 ns & 100.000 ns \\
$10^1$x$10^1$ & 5\% & 5 & 152.800 ns & 264.850 ns & 111.250 ns \\
$10^1$x$10^1$ & 10\% & 10 & 161.550 ns & 231.300 ns & 146.700 ns \\
$10^1$x$10^1$ & 20\% & 20 & 154.150 ns & 222.750 ns & 212.850 ns \\
$10^2$x$10^2$ & 1\% & 100 & 207.950 ns & 2.588 µs & 1.138 µs \\
$10^2$x$10^2$ & 5\% & 500 & 187.700 ns & 2.539 µs & 2.115 µs \\
$10^2$x$10^2$ & 10\% & 1000 & 142.000 ns & 2.740 µs & 4.144 µs \\
$10^2$x$10^2$ & 20\% & 2000 & 233.450 ns & 2.488 µs & 13.849 µs \\
$10^3$x$10^3$ & 1\% & 10000 & 635.300 ns & 80.230 µs & 65.414 µs \\
$10^3$x$10^3$ & 5\% & 50000 & 858.350 ns & 89.192 µs & 223.284 µs \\
$10^3$x$10^3$ & 10\% & 100000 & 844.100 ns & 91.729 µs & 672.044 µs \\
$10^3$x$10^3$ & 20\% & 200000 & 717.250 ns & 97.935 µs & 1.295 ms \\
$10^4$x$10^4$ & 0.0001\% & 100 & 146.550 ns &  & 545.250 ns \\
$10^4$x$10^4$ & 0.001\% & 1000 & 135.200 ns &  & 4.324 µs \\
$10^4$x$10^4$ & 0.01\% & 10000 & 302.050 ns &  & 44.709 µs \\
$10^5$x$10^5$ & 1e-05\% & 1000 & 214.450 ns &  & 6.978 µs \\
$10^5$x$10^5$ & 0.0001\% & 10000 & 498.400 ns &  & 57.752 µs \\
$10^5$x$10^5$ & 0.001\% & 100000 & 745.450 ns &  & 648.069 µs \\
$10^6$x$10^6$ & 1e-06\% & 10000 & 589.350 ns &  & 43.636 µs \\
$10^6$x$10^6$ & 1e-05\% & 100000 & 1.080 µs &  & 526.994 µs \\
$10^6$x$10^6$ & 0.0001\% & 1000000 & 4.596 ms &  & 5.626 ms \\
\bottomrule
\end{tabular}

\pagebreak

\subsection{Set}
Possui as mesmas características que Get.


\begin{tabular}{c c c c c c}
\toprule
Tamanho & Ocupação & population & HashMapMatrix & TableMatrix & TreeMatrix \\
\midrule
$10^1$x$10^1$ & 1\% & 1 & 168.000 ns & 286.450 ns & 663.250 ns \\
$10^1$x$10^1$ & 5\% & 5 & 137.250 ns & 238.650 ns & 118.700 ns \\
$10^1$x$10^1$ & 10\% & 10 & 150.500 ns & 232.150 ns & 186.900 ns \\
$10^1$x$10^1$ & 20\% & 20 & 149.100 ns & 300.750 ns & 383.150 ns \\
$10^2$x$10^2$ & 1\% & 100 & 198.050 ns & 3.444 µs & 921.800 ns \\
$10^2$x$10^2$ & 5\% & 500 & 145.250 ns & 2.602 µs & 2.817 µs \\
$10^2$x$10^2$ & 10\% & 1000 & 142.500 ns & 3.583 µs & 6.031 µs \\
$10^2$x$10^2$ & 20\% & 2000 & 199.050 ns & 2.659 µs & 11.045 µs \\
$10^3$x$10^3$ & 1\% & 10000 & 756.550 ns & 80.424 µs & 46.669 µs \\
$10^3$x$10^3$ & 5\% & 50000 & 777.250 ns & 102.121 µs & 231.141 µs \\
$10^3$x$10^3$ & 10\% & 100000 & 846.000 ns & 94.056 µs & 525.222 µs \\
$10^3$x$10^3$ & 20\% & 200000 & 753.150 ns & 102.774 µs & 1.107 ms \\
$10^4$x$10^4$ & 0.0001\% & 100 & 147.700 ns &  & 1.195 µs \\
$10^4$x$10^4$ & 0.001\% & 1000 & 207.500 ns &  & 8.130 µs \\
$10^4$x$10^4$ & 0.01\% & 10000 & 251.600 ns &  & 45.633 µs \\
$10^5$x$10^5$ & 1e-05\% & 1000 & 211.350 ns &  & 5.151 µs \\
$10^5$x$10^5$ & 0.0001\% & 10000 & 380.750 ns &  & 48.725 µs \\
$10^5$x$10^5$ & 0.001\% & 100000 & 775.800 ns &  & 538.387 µs \\
$10^6$x$10^6$ & 1e-06\% & 10000 & 715.550 ns &  & 50.742 µs \\
$10^6$x$10^6$ & 1e-05\% & 100000 & 992.050 ns &  & 541.520 µs \\
$10^6$x$10^6$ & 0.0001\% & 1000000 & 4.347 ms &  & 5.530 ms \\
\bottomrule
\end{tabular}


\pagebreak
\subsection{Transpor}

Transposição é possível considerar que é afetado pela liberação de memoria por causa que,

\begin{enumerate}
    \item Em HashSet possui um tempo aproximadamente constante, por causa que o HashSet possui apenas um bloco gigante de memoria, então é preciso liberar apenas bloco.
    \item Em TableMatrix é afetado porque são $10^3+1$ blocos de memorias de tamanho $10^3$  para serem liberados. 
    \item Em TreeMatrix são diversos blocos de memorias da BTree que precisam ser liberados.
\end{enumerate}
Essa me parece ser a melhor explicação para isso estar acontecendo, tentei implementar os testes sem ser medido a liberação de memoria mas não consegui.

\begin{tabular}{c c c c c c}
\toprule
Tamanho & Ocupação & population & HashMapMatrix & TableMatrix & TreeMatrix \\
\midrule
$10^1$x$10^1$ & 1\% & 1 & 103.250 ns & 1.217 µs & 70.100 ns \\
$10^1$x$10^1$ & 5\% & 5 & 88.450 ns & 1.106 µs & 79.050 ns \\
$10^1$x$10^1$ & 10\% & 10 & 80.550 ns & 1.073 µs & 116.200 ns \\
$10^1$x$10^1$ & 20\% & 20 & 78.700 ns & 1.121 µs & 178.250 ns \\
$10^2$x$10^2$ & 1\% & 100 & 125.200 ns & 37.662 µs & 620.900 ns \\
$10^2$x$10^2$ & 5\% & 500 & 148.900 ns & 31.996 µs & 2.173 µs \\
$10^2$x$10^2$ & 10\% & 1000 & 179.250 ns & 31.546 µs & 4.720 µs \\
$10^2$x$10^2$ & 20\% & 2000 & 227.700 ns & 30.339 µs & 10.673 µs \\
$10^3$x$10^3$ & 1\% & 10000 & 521.250 ns & 9.856 ms & 90.702 µs \\
$10^3$x$10^3$ & 5\% & 50000 & 659.200 ns & 10.350 ms & 227.072 µs \\
$10^3$x$10^3$ & 10\% & 100000 & 777.150 ns & 10.190 ms & 566.442 µs \\
$10^3$x$10^3$ & 20\% & 200000 & 514.800 ns & 10.782 ms & 1.168 ms \\
$10^4$x$10^4$ & 0.0001\% & 100 & 83.500 ns &  & 551.750 ns \\
$10^4$x$10^4$ & 0.001\% & 1000 & 89.250 ns &  & 4.212 µs \\
$10^4$x$10^4$ & 0.01\% & 10000 & 172.600 ns &  & 52.106 µs \\
$10^5$x$10^5$ & 1e-05\% & 1000 & 106.550 ns &  & 4.241 µs \\
$10^5$x$10^5$ & 0.0001\% & 10000 & 176.900 ns &  & 45.456 µs \\
$10^5$x$10^5$ & 0.001\% & 100000 & 538.000 ns &  & 602.392 µs \\
$10^6$x$10^6$ & 1e-06\% & 10000 & 462.400 ns &  & 47.406 µs \\
$10^6$x$10^6$ & 1e-05\% & 100000 & 538.650 ns &  & 560.323 µs \\
$10^6$x$10^6$ & 0.0001\% & 1000000 & 3.309 ms &  & 6.149 ms \\
\bottomrule
\end{tabular}


\pagebreak
\subsection{Multiplicação Escalar}
Nesse caso, claramente é perceptivel, que para TableMatrix, depende puramente do tamanho da matriz e não de quantos elementos possuem na matriz. Onde no caso $10^3\times10^3$ com ocupação menor que 10\% é TableMatrix pior que os outros casos, já com 20\% é melhor que TreeMatrix. 

\begin{tabular}{c c c c c c}
\toprule
Tamanho & Ocupação & population & HashMapMatrix & TableMatrix & TreeMatrix \\
\midrule
$10^1$x$10^1$ & 1\% & 1 & 173.500 ns & 1.273 µs & 193.350 ns \\
$10^1$x$10^1$ & 5\% & 5 & 167.800 ns & 1.111 µs & 202.650 ns \\
$10^1$x$10^1$ & 10\% & 10 & 277.800 ns & 1.125 µs & 257.400 ns \\
$10^1$x$10^1$ & 20\% & 20 & 324.500 ns & 1.140 µs & 601.300 ns \\
$10^2$x$10^2$ & 1\% & 100 & 875.550 ns & 42.736 µs & 2.203 µs \\
$10^2$x$10^2$ & 5\% & 500 & 3.936 µs & 42.345 µs & 11.389 µs \\
$10^2$x$10^2$ & 10\% & 1000 & 7.617 µs & 42.840 µs & 21.201 µs \\
$10^2$x$10^2$ & 20\% & 2000 & 15.423 µs & 39.586 µs & 45.983 µs \\
$10^3$x$10^3$ & 1\% & 10000 & 76.033 µs & 3.749 ms & 213.259 µs \\
$10^3$x$10^3$ & 5\% & 50000 & 469.930 µs & 3.019 ms & 1.216 ms \\
$10^3$x$10^3$ & 10\% & 100000 & 961.508 µs & 3.020 ms & 2.533 ms \\
$10^3$x$10^3$ & 20\% & 200000 & 2.280 ms & 3.017 ms & 5.306 ms \\
$10^4$x$10^4$ & 0.0001\% & 100 & 1.552 µs &  & 2.418 µs \\
$10^4$x$10^4$ & 0.001\% & 1000 & 11.062 µs &  & 24.042 µs \\
$10^4$x$10^4$ & 0.01\% & 10000 & 105.187 µs &  & 217.544 µs \\
$10^5$x$10^5$ & 1e-05\% & 1000 & 7.622 µs &  & 22.811 µs \\
$10^5$x$10^5$ & 0.0001\% & 10000 & 80.119 µs &  & 210.217 µs \\
$10^5$x$10^5$ & 0.001\% & 100000 & 994.864 µs &  & 2.631 ms \\
$10^6$x$10^6$ & 1e-06\% & 10000 & 104.900 µs &  & 210.487 µs \\
$10^6$x$10^6$ & 1e-05\% & 100000 & 1.076 ms &  & 2.838 ms \\
$10^6$x$10^6$ & 0.0001\% & 1000000 & 17.700 ms &  & 26.585 ms \\
\bottomrule
\end{tabular}

\pagebreak
\subsection{ Adição}
Para adição, o TableMatrix se demonstra muito mais eficiente do que as duas outras estruturas, sendo pior apenas nos casos 1\%. Apesar da quantidade de elementos TableMatrix ser maior que as outras estruturas e elementos que precisam ser lidos também. Processadores são extremamente otimizados para leitura e escrita sequencial de elementos, que faz com que TableMatrix se torne mais eficiente.  

\begin{tabular}{c c c c c c}
\toprule
Tamanho & Ocupação & population & HashMapMatrix & TableMatrix & TreeMatrix \\
\midrule
$10^1$x$10^1$ & 1\% & 1 & 644.550 ns & 886.950 ns & 445.750 ns \\
$10^1$x$10^1$ & 5\% & 5 & 848.200 ns & 755.650 ns & 709.400 ns \\
$10^1$x$10^1$ & 10\% & 10 & 1.489 µs & 766.650 ns & 1.699 µs \\
$10^1$x$10^1$ & 20\% & 20 & 3.319 µs & 770.750 ns & 3.854 µs \\
$10^2$x$10^2$ & 1\% & 100 & 11.958 µs & 39.450 µs & 23.220 µs \\
$10^2$x$10^2$ & 5\% & 500 & 80.782 µs & 40.411 µs & 141.043 µs \\
$10^2$x$10^2$ & 10\% & 1000 & 141.338 µs & 42.648 µs & 286.892 µs \\
$10^2$x$10^2$ & 20\% & 2000 & 253.380 µs & 35.940 µs & 511.524 µs \\
$10^3$x$10^3$ & 1\% & 10000 & 1.328 ms & 4.213 ms & 2.333 ms \\
$10^3$x$10^3$ & 5\% & 50000 & 9.592 ms & 3.937 ms & 12.612 ms \\
$10^3$x$10^3$ & 10\% & 100000 & 18.137 ms & 3.811 ms & 26.052 ms \\
$10^3$x$10^3$ & 20\% & 200000 & 44.696 ms & 3.782 ms & 55.186 ms \\
$10^4$x$10^4$ & 0.0001\% & 100 & 12.680 µs &  & 19.994 µs \\
$10^4$x$10^4$ & 0.001\% & 1000 & 140.318 µs &  & 205.606 µs \\
$10^4$x$10^4$ & 0.01\% & 10000 & 1.348 ms &  & 2.252 ms \\
$10^5$x$10^5$ & 1e-05\% & 1000 & 147.205 µs &  & 208.942 µs \\
$10^5$x$10^5$ & 0.0001\% & 10000 & 1.307 ms &  & 2.536 ms \\
$10^5$x$10^5$ & 0.001\% & 100000 & 16.764 ms &  & 26.623 ms \\
$10^6$x$10^6$ & 1e-06\% & 10000 & 2.129 ms &  & 2.311 ms \\
$10^6$x$10^6$ & 1e-05\% & 100000 & 26.515 ms &  & 30.957 ms \\
$10^6$x$10^6$ & 0.0001\% & 1000000 & 392.212 ms &  & 296.280 ms \\
\bottomrule
\end{tabular}


\subsection{Multiplicação}
Mutiplicação aqui aparece como a operação que se tem o maior proveito da esparcidade da matriz,  como no caso $10^3\times10^3$ e 1\%, onde HashMapMatrix é mais de 100x mais performático que  TableMatrix.

Nese caso também HashMapMatrix e TreeMatrix tem valores muito proximos também, apesar de que HashMapMatrix é constantemente mais performatico.

\begin{tabular}{c c c c c c}
\toprule
Tamanho & Ocupação & population & HashMapMatrix & TableMatrix & TreeMatrix \\
\midrule
$10^1$x$10^1$ & 1\% & 1 & 1.774 µs & 1.688 µs & 930.250 ns \\
$10^1$x$10^1$ & 5\% & 5 & 2.377 µs & 1.760 µs & 2.231 µs \\
$10^1$x$10^1$ & 10\% & 10 & 5.196 µs & 1.894 µs & 4.310 µs \\
$10^1$x$10^1$ & 20\% & 20 & 9.396 µs & 1.807 µs & 11.884 µs \\
$10^2$x$10^2$ & 1\% & 100 & 43.326 µs & 1.481 ms & 54.660 µs \\
$10^2$x$10^2$ & 5\% & 500 & 400.137 µs & 1.725 ms & 1.662 ms \\
$10^2$x$10^2$ & 10\% & 1000 & 1.187 ms & 1.594 ms & 2.486 ms \\
$10^2$x$10^2$ & 20\% & 2000 & 4.040 ms & 1.632 ms & 8.941 ms \\
$10^3$x$10^3$ & 1\% & 10000 & 13.442 ms & 1.946 s & 31.565 ms \\
$10^3$x$10^3$ & 5\% & 50000 & 509.380 ms & 1.896 s & 882.779 ms \\
$10^3$x$10^3$ & 10\% & 100000 & 2.175 s & 1.966 s & 3.271 s \\
$10^3$x$10^3$ & 20\% & 200000 & 8.239 s & 2.184 s & 9.949 s \\
$10^4$x$10^4$ & 0.0001\% & 100 & 35.514 µs &  & 53.087 µs \\
$10^4$x$10^4$ & 0.001\% & 1000 & 420.390 µs &  & 555.832 µs \\
$10^4$x$10^4$ & 0.01\% & 10000 & 4.302 ms &  & 7.091 ms \\
$10^5$x$10^5$ & 1e-05\% & 1000 & 355.582 µs &  & 730.562 µs \\
$10^5$x$10^5$ & 0.0001\% & 10000 & 4.135 ms &  & 5.417 ms \\
$10^5$x$10^5$ & 0.001\% & 100000 & 76.117 ms &  & 91.635 ms \\
$10^6$x$10^6$ & 1e-06\% & 10000 & 4.191 ms &  & 5.961 ms \\
$10^6$x$10^6$ & 1e-05\% & 100000 & 85.859 ms &  & 88.136 ms \\
$10^6$x$10^6$ & 0.0001\% & 1000000 & 1.221 s &  & 1.515 s \\
\bottomrule
\end{tabular}



\section{Analise Assintótica}

Foi gerado graficos, onde o eixo X é a quantidade de elementos que cada matriz possui, e o eixo Y é a razão da duração do sample por alguma função assintótica.

Isso é usado por causa de que:
$$\lim_{n\to \infty} \frac{f(x)}{g(x) } \in \mathbb{R}\to f(x) \in \Theta(x)$$
$$\lim_{n\to \infty} \frac{f(x)}{g(x) } = \infty \to f(x) \in \Omega(g(x))$$
$$\lim_{n\to \infty} \frac{f(x)}{g(x) } = 0 \to f(x) \in O(g(x))$$

Portanto, dado isso, podemos definir então por cima, de que caso a razão dos samples estiver tendendo 0 então é $O$, caso esteja crescente então é $\Omega$, caso esteja nem decrescendo indefinidamente nem crescendo indefinidamente então é $\Theta$ 

Nos graficos há tres partes:
\begin{enumerate}
    \item Curva do Supremo: Curva pontos $(k, d)$ que possuem a propriedade $\forall (k', d'), k> k' \to d > d'$, portanto o ponto que é supremo dos valores a partir dele para frente.
    \item Curva do Infimo: Curva dos pontos $(k, d)$ que possuem a propriedade $\forall (k', d'), k> k' \to d < d'$, portanto o ponto que é o infimo dos valores a partir dele para frente.

    \item Pontos Samples coloridos: Os pontos que foram obtidos do experimento, onde são coloridos com base na ocupação. 
    \item Linha media ponderada: Linha horizontal da media ponderada da duração, onde o peso de cada ponto é a população. 
\end{enumerate}

\subsection{TreeMapMatrix}

\subsubsection{set}

Para modificar um valor, tem tempo, que a razão tende a crescer um pouco a mais que $\log k$, porém, extremamente inferior a linear, demonstrando que tem uma aproximação ao estimado que é $\Theta(\log k)$  
\duasfiguras
    {graficos/TreeMatrix/set/1-log_matrix_performance.png}
    {$\log k$}
    {graficos/TreeMatrix/set/2-linear_matrix_performance.png}
    {$k$}
\pagebreak
\subsubsection{get}

Para acesso, as mesmas caracteristicas que set. 
\duasfiguras
    {graficos/TreeMatrix/get/1-log_matrix_performance.png}
    {$\log k$}
    {graficos/TreeMatrix/get/2-linear_matrix_performance.png}
    {$k$}

\subsubsection{Transposição}
O comportamento da transposição tende a parecer linear ao crescer, mas demonstra uma taxa de crescimento.
\duasfiguras
    {graficos/TreeMatrix/transpose/0-constant_matrix_performance.png}
    {$1$}
    {graficos/TreeMatrix/transpose/1-log_matrix_performance.png}
    {$k$}

    
\subsubsection{Multiplicação Escalar}
O grafico se demonstra crescente em relação a $\log k$ e decrescente em relação a $k$, portanto é uma complexidade pior que logaritmica, mas melhor que linear 
\duasfiguras
    {graficos/TreeMatrix/muls/1-log_matrix_performance.png}
    {$\log k$}
    {graficos/TreeMatrix/muls/2-linear_matrix_performance.png}
    {$k$}
\pagebreak
\subsubsection{Adição}
O comportamento para soma, tem uma tendencia extremamente alta para convergir na razão pelo linear, mostrando um comportamento claramente na parte testada $\Theta(k)$, melhor do que o calculado.
\duasfiguras
    {graficos/TreeMatrix/add/2-linear_matrix_performance.png}
    {$k$}
    {graficos/TreeMatrix/add/3-nlog_matrix_performance.png}
    {$k\log k$}
    
\subsubsection{Multiplicação Matrizes}

Para multiplicação de matrizes, o algoritmo demonstra ter um comportamento próximo de $k \log k \sqrt k$  porém com uma tendencia de crescimento, porém com uma tendencia decrescente em relação a $k^2$.

\duasfiguras
    {graficos/TreeMatrix/mul/5-nlogsqrt_matrix_performance.png}
    {$k \log k \sqrt k$}
    {graficos/TreeMatrix/mul/6-quadratic_matrix_performance.png}
    {$k^2$}



\pagebreak

\subsection{HashMapMatrix}

\subsubsection{set}

Para o set, apesar do esperado ter sido estimado como constante, as métricas tenderam  para estar se aproximando em relação a $\log k$, e ser muito menor que linear.
\duasfiguras
    {graficos/HashMapMatrix/set/1-log_matrix_performance.png}
    {$\log k$}
    {graficos/HashMapMatrix/set/2-linear_matrix_performance.png}
    {$k$}
\subsubsection{get}
Para acesso, as mesmo comportamento que set. 
\duasfiguras
    {graficos/HashMapMatrix/get/1-log_matrix_performance.png}
    {$\log k$}
    {graficos/HashMapMatrix/get/2-linear_matrix_performance.png}
    {$k$}
\pagebreak
\subsubsection{Transposição}
O comportamento da transposição está próximo do comportamento do get e do set, mostrando que provavelmente deve ser um problema em relação ao experimento, o fato de estar mais próximo de log do que constante.
\duasfiguras
    {graficos/HashMapMatrix/transpose/1-log_matrix_performance.png}
    {$1$}
    {graficos/HashMapMatrix/transpose/2-linear_matrix_performance.png}
    {$k$}

    
\subsubsection{Multiplicação Escalar}
Na multiplicação de escalar é possivel perceber que tem uma performance suavemente melhor que $k \log k$ e suavemente pior que $k$. Mas além disso um detalhe importante é perceber o comportamente ocilatorio do valor, que se dá ao quanto da tabela é ocupada, onde quando temos uma tabela com a capacidade proxima do tamanho, temos uma melhor performance, e como a ocupação da tabela vai variando, então a performance varia, mas se mantêm aproximadamente a razão.
\duasfiguras
    {graficos/HashMapMatrix/muls/2-linear_matrix_performance.png}
    {$k$}
    {graficos/HashMapMatrix/muls/3-nlog_matrix_performance.png}
    {$k\log k$}
\pagebreak
\subsubsection{Adição}
Tem um comportamento um pouco pior que linear, e um pouco melhor que log-linear, que provavelmente é causado pelas colisões de hash.
\duasfiguras
    {graficos/HashMapMatrix/add/2-linear_matrix_performance.png}
    {$k$}
    {graficos/HashMapMatrix/add/3-nlog_matrix_performance.png}
    {$k\log k$}
    
\subsubsection{Multiplicação Matrizes}

Para multiplicação de matrizes, o algoritmo demonstra ter um comportamento próximo de $k \log k \sqrt k$  porém com uma tendencia de crescimento, porém com uma tendencia decrescente em relação a $k^2$, então como estimado é melhor que $k^2$. 

\duasfiguras
    {graficos/HashMapMatrix/mul/5-nlogsqrt_matrix_performance.png}
    {$k \log k \sqrt k$}
    {graficos/HashMapMatrix/mul/6-quadratic_matrix_performance.png}
    {$k^2$}




\pagebreak  

\subsection{TableMatrix}
Para TableMatrix, a analise assintótica em relação ao quantidade de elementos não nulo, não faz sentido, porque não existe um limite superior de quão ruim um caso pode ser, onde o caso sempre cresce indefinidamente junto ao tamanho da matriz e não em relação a quantidade de elementos não nulos. 

Isso é perceptivel nos graficos de que há uma clara separação na maior parte deles do tempo em relação a cada taxa de ocupação (indicada pelas cores dos pontos).

\begin{figure}[h]
    \centering
    \begin{minipage}{0.45\textwidth}
        \centering
        \includegraphics[width=\linewidth]{graficos/TableMatrix/set/0-constant_matrix_performance.png}
        \caption{Set}
    \end{minipage}
    \hfill
    \begin{minipage}{0.45\textwidth}
        \centering
        \includegraphics[width=\linewidth]{graficos/TableMatrix/get/0-constant_matrix_performance.png}
        \caption{Get}
    \hfill
    \end{minipage}
        \begin{minipage}{0.45\textwidth}
        \centering
        \includegraphics[width=\linewidth]{graficos/TableMatrix/transpose/0-constant_matrix_performance.png}
        \caption{Transpor}
    \end{minipage}
    \hfill
    \begin{minipage}{0.45\textwidth}
        \centering
        \includegraphics[width=\linewidth]{graficos/TableMatrix/muls/0-constant_matrix_performance.png}
        \caption{Multiplicação escalar}
    \end{minipage}
    \hfill
    \begin{minipage}{0.45\textwidth}
        \centering
        \includegraphics[width=\linewidth]{graficos/TableMatrix/add/0-constant_matrix_performance.png}
        \caption{Adição}
    \end{minipage}
    \hfill
    \begin{minipage}{0.45\textwidth}
        \centering
        \includegraphics[width=\linewidth]{graficos/TableMatrix/mul/0-constant_matrix_performance.png}
        \caption{Multiplicação Matriz}
    \end{minipage}
\end{figure}

\pagebreak

Pegando a multiplicação de matriz em destaque é perceptivel, que a linha do supremo segue exatamente uma linha de cores iguais, que é exatamente os pontos do $1\%$ de ocupação, onde é 1\% de matrizes muito maiores que a população, então se diverge das outras.


\begin{figure}[h]
    \centering
    \begin{minipage}{0.6\textwidth}
        \centering
        \includegraphics[width=\linewidth]{graficos/TableMatrix/mul/0-constant_matrix_performance.png}
        \caption{Multiplicação Matriz}
    \end{minipage}
\end{figure}
Mas caso, plotamos os valores, mas ao invés de colocarmos no eixo x a quantidade de elementos não nulos, colocarmos a quantidade de elementos no total, é perceptível que os dados são muito mais correlacionados. 
\begin{figure}[h]
    \centering
    \begin{minipage}{0.6\textwidth}
        \centering
        \includegraphics[width=\linewidth]{graficos/TableMatrix/mul/size_matrix_performance.png}
        \caption{Multiplicação Matriz}
    \end{minipage}
\end{figure}
\pagebreak
\section{Conclusão}

Dado toda analise teórica e medições feitas é possível tirar certas conclusões:

Como a memoria é usada, liberada e feita é extremamente importante, visto que mesmo para a TableMatriz, acesso aleatório piorou com o crescimento  da matriz, por causa de fatores como leitura da memoria, caching de memoria e paginação da memoria pelo sistema operacional (Uma matriz $10^2\times 10^2$ já é maior que 4kb que é maior que uma pagina de sistema). 

Para implementação que não são esparsas é extremamente mais eficiente o uso de matrizes diretas, porque tem um aproveitamento muito melhor da arquitetura do computador.  

HashTableMatrix tem uma eficiência maior no geral do que TreeMapMatrix, porém, ao custo de que pode ter custos imprevisíveis certas operações. Sendo um deles o custo de aumentar o tamanho da HashTable, que apesar de ser amortizado pode acabar levando a interrupções temporárias em sistemas, como por exemplo, caso haja uma matriz esparsa que ocupe 1GB de memoria, e seja necessário aumentar o tamanho dela, isso levaria a ter que começar a fazer uma copia de 1GB de memoria. Problema esse que não existe no TreeMapMatrix, por causa que o crescimento e decrescimento é granular, sem a necessidade de liberação e alocação de quantidades gigantes de memoria.

\subsection{Nota: Abstração}
Um problema que aconteceu durante o desenvolvimento, foi de que foi tentando fazer multiplicação de matrizes de tamanho $10^5\times 10^5$ com 1\% de ocupação porém não foi possível e não foi por causa do tempo de processamento, mas sim por causa por causa de que a matriz resultante da multiplicação tinha uma taxa de ocupação muito maior que fazia com que a matriz resultante e os artefatos intermediários não cabiam na memoria.

Para isso considerei usar resolver esse problema, aproveitando exatamente de como funciona o MapMatrix, sem reescrever a MapMatrix, e implementar uma BTree persistida no disco, no qual não iria afetar de forma tão grande o desempenho por ser no disco, por causa que a implementação de MapMatrix de multiplicação e soma (mais custosos), usa iteradores que faz com que para iterar na matriz, bastasse ler cada bloco de disco para memoria, iterar sobre eles, fazendo as operações sequencialmente. Fazendo assim leitura do disco em grandes blocos e de forma sequencial algo extremamente mais eficiente do que acesso aleatório do disco.

Isso sem precisar modificar o funcionamento do MapMatrix, já que é feito sobre um dos pilares da programação a \textbf{abstração}.


% \begin{align*}
% \sum_{c\in A_C}\left( \log |B_L| + \sum_{a\in A}\sum_{b\in B} \mathbf{1}_{a_c = c}\cdot\mathbf{1}_{c=b_l} \right)
% \\|A_C| \log |B_L|+\sum_{a\in A} \left(\sum_{b\in B} \mathbf{1}_{a_c = b_l} \right)
% \\|A_C|\log |B_L|  + \sum_{a\in A}\sum_{b\in B} \mathbf{1}_{a_c = b_l} 
% \\|A_C|\log |B_L|  + \sum_{(a, b)\in A\times B} \mathbf{1}_{a_c = b_l} 
% \\ \sum _{i=1}^n \left(\sum_{a\in A} \mathbf{1}_{a_c = i} \right)  \left(\sum_{b\in B} \mathbf{1}_{b_l = i} \right)  
% \end{align*}

% \begin{align*}
%     \\a_j' = a_j + \delta_a
% \\a_k' = a_k -\delta_ a
% \\
% \\b_j' = b_j + \delta_b
% \\b_k' = b_k -\delta_b
% \\\Delta =  a_j'b_j' + a_k'b_k' -(a_jb_j + a_kb_k) 
% \\\Delta =  (a_j + \delta_a)b_j' + (a_k - \delta_a)b_k' - a_jb_j - a_kb_k
% \\\Delta =  a_jb_j' + \delta_ab_j' + a_kb_k' - \delta_a b_k'  - a_jb_j - a_kb_k
% \\\Delta =  a_j(b_j' - b_j) + a_k(b_k'-b_k) + \delta_a (b_j' - b_k')
% \\\Delta =  a_j(\delta_b) + a_k(-\delta_b) + \delta_a (b_j + \delta _b - (b_k-\delta_b) )
% \\\Delta =  \delta_b(a_j- a_k) + \delta_a (b_j - b_k +2\delta_b) 
% \\\Delta =  \delta_b(a_j- a_k) + \delta_a (b_j - b_k)  2\delta_a\delta_b 
% \end{align*}

% \begin{align*}
% a'_i = \begin{cases}
% a_i+\delta &,\text{se } i=j
% \\ a_i - \delta &, \text{se } i = k
% \\ a_i &,\text{se não}  
% \end{cases}
% \\a_j'+ a_k '= a_j + a_k
% \\\sum _{i=1}^n a_iy_i <  \sum _{i=1}^n a_i'y_i
% \\ a_jy_j + a_ky_k <  a_j'y_j + a_k'y_k  
% \\ a_jy_j + a_ky_k <  (a_j + \delta)y_j + (a_k - \delta)y_k  
% \\a_jy_j + a_ky_k < a_jy_j + \delta y_j + a_ky_k - \delta y_k  
% \\0<  \delta (y_j - y_k)
% \\1=\text{sign}(\delta (y_j - y_k))  
% \\1=\text{sign}(\delta) \text{sign}(y_j - y_k)  
% \\\text{sign}(\delta) =\text{sign}(y_j - y_k)  
% \end{align*}



% $$m = \frac{|A|(1- d_B )}{d_B (1-c)}$$
% $$k =  \frac{|A|( d_B - c )}{d_B (1-c)}$$

\end{document}
